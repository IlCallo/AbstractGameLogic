\chapter{Introduction}
	
	The first thing you think when the words “mobile” and “augmented reality” are put together, is probably Pokémon GO. This is thanks to the unexpected popularity which followed its release early this year, but thinking about its game mechanics, it is noticeable that the Augmented Reality concept is heavily underused there and could be exploited a lot better.
	Some others games exploit this potentiality more, like Father.IO, but require some kind of physical add-on to play. \\
	
	In recent years, the mobile games market is guided by some dogmas (with some minor exceptions):
	\begin{itemize}
		\item the game is free;
		\item the game has an in-game shop to progress quickly, creating a pay-per-win environment;
		\item the game is mostly a single user experience;
		\item the user should be able to play the game whenever and wherever he wants;
		\item the user is induced to play the game as much as possible: the more it is addicting, higher the probability that he will pay for additional contents.
	\end{itemize}
	
	This thesis, knowing that such schema is the most used and accepted, will explore the exact opposite direction to see if something different exists and trying to insert an AR layer which is really meaningful for the game, instead of being just some kind of good-looking add-on. \\
	
	The objective is to design an AR game which: 
	\begin{itemize}
		\item can be installed and used only by a small group of people at a time;
		\item does not provide a pay-per-win environment and focus on other monetization system;
		\item focus on playing as a coordinated team, while giving enough decision power to every player;
		\item can be accessed and used only in a specific area and for a limited amount of time;
		\item does not try to keep the user on it, instead the limited amount of time he is given shall increase the value of the game.
	\end{itemize}
	
	Given these bases, the most interesting field to explore is the one of structured recreational events: an experience limited in time (usually a couple of hours) for which it is paid a participation fee in exchange of a new and, hopefully, funny kind of game which could also represent a challenge against yourself or other players.
	
	To narrow the scope, examples of this kind of events are the Zombie Runs, Room Escapes and, partially, paint-ball games. These events built a successful business model while following all the points previously stated.
	My project starts by some of their concepts and uses them on a city-wide area using the AR potentiality to take the game to physical locations. \\
	
	Apart from this kind of events, this work has been influenced by a lot of other sources consulted during the preparation phases and even later, during the development of the project.
	It takes some hints from existing AR games (Pokémon GO, Ingress, Father.io), especially regarding what \emph{not} to do.
	It shows similarities to GeoGuild\cite{ionescu:geoguild} framework, even if such product was discovered when the game design was already ongoing and it is a confirmation of the chosen path more than a influencer.
	My work has been heavily influenced both from role-playing games, especially in their \emph{LARP}\footnote{A live action role-playing game (LARP) is a form of role-playing game where the participants physically act out their characters' actions. The players pursue goals within a fictional setting represented by the real world while interacting with each other in character. The outcome of player actions may be mediated by game rules or determined by consensus among players. Event arrangers called game masters decide the setting and rules to be used and facilitate play.\cite{wiki:larp}} form, and from \emph{Eurogames}\footnote{A Eurogame, also called German-style board game, German game, or Euro-style game, is any of a class of tabletop games that generally have indirect player interaction and abstract physical components. Such games emphasize strategy, downplay luck and conflict, lean towards economic rather than military themes, and usually keep all the players in the game until it ends.\cite{wiki:eurogame}}, in particular \textbf{Discworld: Ankh Morpork} that constitutes the base from which the project's game mechanics had been derived and adapted to the AR environment.
	
	The project development cycle had been divided in four phases: rule definition, implementation, graphics and setting definition, testing.
	
	\textbf{Unluckily, the testing phase has not taken place, yet, due to time constraints.}