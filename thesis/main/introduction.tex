\chapter{Introduction}
	
	The first thing you think when the words "mobile" and "augmented reality" are put together, is probably Pokémon GO. This is thanks to the unexpected popularity which followed his release early this year, but thinking a little about its game mechanics, it's noticeable that the Augmented Reality concept is heavily underused there and could be exploited a lot better.
	Some others games exploit more this potentiality, like Father.IO, but require some kind of physical add-on to play. \\
	
	In recent years, the mobile games market is guided by some dogmas (with some minor exceptions):
	\begin{itemize}
		\item the game is be freely installable;
		\item the game have an in-game shop to progress quickly, creating a pay-per-win environment;
		\item the game is mostly on the single user playing;
		\item the user is be able to access the game whenever and wherever he wants;
		\item the user is induced to play to the game as much as possible: the more it's addicting, the higher the probability that he'll pay for additional contents.
	\end{itemize}
	
	This thesis, knowing that such schema is the most used and accepted, will explore the exact opposite direction to see if something different exists and trying to insert an AR layer which is really meaningful for the game, instead of being just some kind of good-locking add-on. \\
	
	The objective is to design an AR game which: 
	\begin{itemize}
		\item can be installed and used only by a small group of people at a time;
		\item does not provide a pay-per-win environment and focus on other monetization system;
		\item focus on playing as a coordinated team, while giving enough decision power to every player;
		\item can be accessed and used only in a specific area and for a limited amount of time;
		\item does not try to keep the user on it, instead the limited amount of time he's given shall increase the value of the game.
	\end{itemize}
	
	Given these basis, the most interesting field to explore is the one of structured recreational events: an experience limited in time (usually a couple of hours) for which you pay a participation fee in exchange of a new and, hopefully, funny kind of game which could also represent a challenge against yourself or other players.
	
	To narrow the scope, examples of this kind of events are the Zombie Runs, Room Escapes and, partially, paint-ball games. This events built a successful business model while following all the points previously stated.
	My project start by some of their concepts and use them on a city-wide area using the AR potentiality to bring the game in physical locations. \\
	
	Apart from those kind of events, this work has been influenced by a lot of other sources consulted during the preparation phases and even later, during the development of the project.
	It takes some hints from already existing AR games (Pokémon GO, Ingress, Father.io), especially on what \emph{not} to do.
	It shows similarities to GeoGuild\cite{ionescu:geoguild} framework, even if it was discovered when the game design was already ongoing and is more a confirm of the chosen path instead of an influencer.
	It had been heavily influenced both from role playing games, especially in their \emph{LARP}\footnote{A live action role-playing game (LARP) is a form of role-playing game where the participants physically act out their characters' actions. The players pursue goals within a fictional setting represented by the real world while interacting with each other in character. The outcome of player actions may be mediated by game rules or determined by consensus among players. Event arrangers called game masters decide the setting and rules to be used and facilitate play.\cite{wiki:larp}} form, and from \emph{Eurogames}\footnote{A Eurogame, also called German-style board game, German game, or Euro-style game, is any of a class of tabletop games that generally have indirect player interaction and abstract physical components. Such games emphasize strategy, downplay luck and conflict, lean towards economic rather than military themes, and usually keep all the players in the game until it ends.\cite{wiki:eurogame}}, in particular \textbf{Discworld: Ankh Morpork} that constituted the base from which the project's game mechanics has been derived and adapted to the AR environment.
	
	The project development cycle has been divided in four phases: rule definition, implementation, graphics and setting definition, testing.
	
	\textbf{Unluckily, the testing phase could not take place due to time constraints.}