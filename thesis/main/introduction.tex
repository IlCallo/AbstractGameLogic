\chapter{Introduction}
	
	The first thing that will pop to your mind when the words "mobile" and "augmented reality" are put together, will probably be Pokémon GO. This is thanks to the unexpected popularity which followed his release early this year, but thinking a little about its game mechanics, we'll notice that actual Augmented Reality layer is heavily underused and could be exploited a lot better.
	Some others games exploit more this potentiality, like Father.io, but require some kind of add-on to play.
	
	In recent years, some of this AR games entered a market dictated by some dogmas (with some minor exceptions):
	\begin{itemize}
		\item the game is freely and easily installable;
		\item the game has an in-game shop to progress quickly;
		\item the game is on the user alone;
		\item the user must be able to access the game whenever and wherever he wants;
		\item the user must be induced to play to the game as much as possible, to increment profitability.
	\end{itemize}
	
	This thesis, knowing that such schema is the most used and accepted, will explore the exact opposite direction to see if something good can happen while travelling there.
	
	The objective is then to design an AR game which: 
	\begin{itemize}
		\item can be installed and used only by a small group of people;
		\item does not provide a pay-per-win environment and focus on other monetization system;
		\item gives a lot of weight on playing as a coordinated team instead of alone;
		\item can be accessed and used only in a specific area and for a limited amount of time;
		\item does not try to keep the user on it, because the user actually \emph{wants} to keep playing it for the limited amount of time he's given.
	\end{itemize}
	
	Given these basis, the field I decide to venture in is the one of structured recreational events: an experience limited in time (usually a couple of hours) for which you pay a fee in exchange of a new and, hopefully, funny sort of game which also represent a challenge versus yourself or versus other players.
	To narrow the scope, examples of this kind of events are the Zombie Runs, Room Escapes and, partially, paint-ball games which have constructed a successful business model while following all the points previously stated.
	My project start by some of their concepts and use them on a city-wide area using the AR potentiality to bring the game in physical locations.
	
	Apart from those kind of events, this work has been influenced by a lot of other sources consulted during the preparation phases and even later, during the development of the project.
	It takes some hints from already existing AR games (Pokémon GO, Ingress, Father.io), especially on what \emph{not} to do.
	It shows similarities to GeoGuild\cite{ionescu:geoguild} framework, even if it was discovered when the game design was already ongoing and is more a confirm of the chosen path instead of an influencer, but the aims and application fields are different.
	It had been heavily influenced both from role playing games, especially in their \emph{LARP}\footnote{A live action role-playing game (LARP) is a form of role-playing game where the participants physically act out their characters' actions. The players pursue goals within a fictional setting represented by the real world while interacting with each other in character. The outcome of player actions may be mediated by game rules or determined by consensus among players. Event arrangers called game masters decide the setting and rules to be used and facilitate play.\cite{wiki:larp}} form, and from \emph{Eurogames}\footnote{A Eurogame, also called German-style board game, German game, or Euro-style game, is any of a class of tabletop games that generally have indirect player interaction and abstract physical components. Such games emphasize strategy, downplay luck and conflict, lean towards economic rather than military themes, and usually keep all the players in the game until it ends.\cite{wiki:eurogame}}, in particular \textbf{Discworld: Ankh Morpork} that constituted the base from which game rules of the project has been derived and adapted to the AR environment.
	
	The project development cycle has been divided in four phases: rule definition, application and server implementation, graphics and setting definition, testing.
	Unluckily, the testing phase could not take place due to time constraints.