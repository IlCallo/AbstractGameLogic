\chapter{State of the art}

	\section{Augmented Reality}
	
		\begin{quoting}
			Augmented reality (AR) is a live direct or indirect view of a physical, real-world environment whose elements are augmented (or supplemented) by computer-generated sensory input such as sound, video, graphics or GPS data.~\cite{wiki:ar}
		\end{quoting}
		
		The first known ideas similar to AR concept go back to 1901 in a novel for boys~\cite{baum:master}, which imagined a set of spectacles capable of showing a label over people heads telling if they were good, evil, foolish, kind, etc.
		Except for the character recognition, this description is no far from a pair of Google Glasses recognizing people profiles through the camera and attaching a label on their heads.
		
		This example is only one of the various techniques on which are based AR applications nowadays.
		
		AR is already used in many fields, in particular in industrial design, medical care and mostly in military.
		
		\subsection{Marker Based}
			
			Based on geometric markers (like QR codes) placed in the space to be augmented, the algorithms that power this technology are capable of recognizing the marker and replace it on the final image stream with a 3D object.
			This was the first attempt of realizing the AR concept.
			
			% Insert image
		
		\subsection{Location Based}
		
			Based mainly on GPS and many other spatial sensors (accelerometer, gyroscope, etc.), this technology is the milestone that let many developers produce AR games located in the real world.
			The main challenges for location-based apps had been the accuracy degree that was possible to achieve and the battery drain consequent to extended use of GPS, but now the Google FusedLocationProviderApi, on Android, standardized the location retrieval exploiting together GPS, nearby Wi-Fi hotspot and cellular network, getting from 2-5 meters of accuracy, and managing in an efficient way the requests in order to save battery charge. % NEEDQUOTE

		\subsection{Markerless Tracker}
			
			\begin{quoting}
				In its simplest form, tracking can be defined as the problem of estimating the trajectory of an object in the image plane as it moves around a scene. In other words, a tracker assigns consistent labels to the tracked objects in different frames of a video. Additionally, depending on the tracking domain, a tracker can also provide object-centric information, such as orientation, area, or shape of an object.~\cite{ylmaz:tracking}
			\end{quoting}
			
			Next generation of AR, which uses algorithms to analyse the environment recorded from the camera and is capable of recognizing and tracking objects. Using this kind of technology, there is no more need to place any marker and the users can interact with the augmented world in a much more effective way.
			
	
	\section{AR In Mobile Devices}
	
		AR main target nowadays is of course mobile devices (both smart phones and smart goggles) given the simultaneous presence of a display and an incorporated camera, united with more than decent processing units.
		
		This is pushing many companies in developing apps and frameworks, while a growing number of IT people start to consider the AR more attractive than VR (Virtual Reality).
		
		The main limitation that AR apps are facing in this field is surely the battery drain that the CPU intensive tasks, GPS usage and enabled data traffic take within themselves, but those limits are being pushed further and further by new generation devices.
		
		Battery duration, as well as other tricks to limit power consumption (in this regard the new A10 processor from Apple with power differentiated cores is only the last of many optimization), soon won't be a problem any more and AR apps will probably bloom on the market.
		
		\subsection{Main Frameworks}
			
			To build up this project, I needed an already developed AR framework to start with, so I analysed which options had been developed in past years searching for some well specified features:
			
			\begin{description}
				\item[cost] free or with a long enough free trial;
				\item[location based] capable of managing AT LEAST location based AR;
				\item[license] under MIT or Apache 2.0, looking forward to use the project with commercial purposes;
				\item[alive] with a strong and passionate community keeping the project alive;
				\item[documentation] good documentation by means of JavaDoc and examples;
				\item[2D] capable of showing AT LEAST 2D pictures in the AR layer.
			\end{description}
			
			Unluckily, I could not find all those qualities in a single framework, so I had to settle with a subset of them.
			
			\subsubsection{Wikitude}
			
				By far, the best AR framework around. It manages 2D pictures and 3D animated objects, it support both native or JavaScript API, supports all 3 main AR technology (location based, marker based and markerless tracking), it's very active, always up to date with last OS and devices, has a good documentation and is distributed for use in commercial projects.
				Only one point against: it's a subscription service.
				A free trial is distributed, but it puts a watermark when using the framework, so we had to let it apart.
			
			\subsubsection{DroidAR}
				
				Supports both marker base and location based AR, has a good documentation with JavaDoc wiki pages and examples, it can manage both 2D and 3D objects.
				This seemed very promising at technical level, but had also a lot of flaws: the project has a open source GNU GPL v3 license, which make it impossible to use for me. Usage under another license is possible, but paying a fee, so it's still not acceptable.
				Adding to this, the project on GitHub has been dead for 4 years now, because the maintainers switched to develop v2, which is closed source.
			
			\subsubsection{Mixare}
			
				Another open source project, another failure. Mixare can only manage 2D objects which consist in a picture or shape and an optional label. The documentation is really poor, and the location based AR seems slow and buggy. It's free of charge, but only under GNU GPL v3 license. Last but not least, the project has been dead for more than 4 years.
				Despite all those problems, the structure of the project, thought in a really modular way, and the ideas behind it is pretty clever, for this i decided to cite it here anyway.
			
			\subsubsection{PanicAR}
			
				This framework seemed especially promising: it is focused on 2D pictures and location based AR and has a decent documentation.
				Unluckily, it's free of charge only for non commercial projects, otherwise a watermark is added, and the Android version, the primary version was for iOs, is dead since 2 years.
			
			\subsubsection{BeyondAR}
			
				Has a really good documentation, tutorials and even a fully functional example app; it's location based and the API is pretty simple and intuitive; it's capable of managing both 2D and 3D objects. It's released under Apache 2.0 free of charge, still the development is open source.
				Unluckily the project is currently dead since 2 years. Despite this, I decided to use this framework in the project, fixing by myself some known bugs and contributing to the development, hoping that the BeyondAR maintainer will come back at some point merging all the pull requests and resurrecting this already pretty good framework.
		
		\subsection{Main AR Games}
		\label{soa:games}
		
			Some pioneers already adventured in the AR world and considering that the main project of this thesis is an AR app, I find relevant to briefly describe the ones who managed to have a decent success, because either they are relevant to show the level AR games reached today or some of their mechanics and idea are similar to the ones I came up to.
		
			\subsubsection{Pokemon GO}
			
				Developed by Niantic in collaboration with Game Freak, Nintendo and The Pokémon Company, has been released to public during 2016 summer and it become quickly a worldwide phenomenon, mostly because of the hype for the background story and not for the technical part laying behind it.
				Classifiable more as an exergame (excercise game, fitness game) than as a Pokémon game, is heavily relying on location based AR (the only function that actually use the device camera is usually turned off by users because it slow down the app and consume more battery) and is substantially just a map where you can see your virtual alter ego and some POIs (Points Of Interest) with which you can interact in a very basic way.
				Tree teams continuously fight each others to gain influence over special places and this promote members of the same team to do it in groups to claim the area ownership faster.
			
			\subsubsection{Ingress}
			
				Developed by Niantic (actually their first AR game), is basically the same game as Pokémon GO, after you strip out the Pokémons and change the background story into a sci-fi one.
				The influence that this game have on Pokémon GO, both in terms of UX (User eXperience) and technology, is pretty strong.
			
			\subsubsection{Life is Crime}
			
				Developed by Red Robot Labs, is a location based RPG (Role Play Game) with a criminal background story. Your neighbourhood become the game board, it's based on concept of gathering with other players living nearby to form a gang and make your reputation rise doing virtual crimes, trying to become the kingpin of your main.zone.
				The software house apparently shut down everything in 2015 while the use of bots and cheats ruined the game.
			
			\subsubsection{Zombie, run!}
			
				Developed by Six to Start and Naomi Alderman, is the first and most famous exergame. Funded on Kickstarter, the game relies on location based AR and is mainly focused on single player and background story, going bucking with respect to other AR games.
				The basic idea is simply to run away from zombies that spawn near you and to listen registrations explaining the story, with sporadic events that brings the community together.
				The app has a seasonal life cycle where new missions, which means also new background story pieces, are released in blocks of 30-40.
				A remarkable feature which has been added let the users set an arrival point and generate an on-the-fly mission which will get you there.
			
			\subsubsection{Parallel Kingdom}
			
				Developed by PerBlue, is the first location based territorial RPG and is based, as many others, in conquering surrounding areas.
				The game is based on a map over which custom markers are placed to identify buildings, monsters and players.
				Both PvE (Player vs Environment) and PvP (Player vs Player) are supported, as well as travel, commerce and many features of classic RPGs.
				It's easily the most complete AR RPG on the market.
			
			\subsubsection{Father.IO}
			
				Developed by Proxy42, is the first FPS (First Person Shooter) using AR techniques to become, as said by the funders, an "Augmented Reality Laser Tag".
				It relies on an add-on to put over the smart phone, called Interceptor, used to fire laser beams at other players and notice theirs incoming.
				The interconnection system offer a good quality, since it combines GPS and local broadcasting to keep the latency low and the shots accuracy high.