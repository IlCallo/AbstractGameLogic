\chapter{Conclusion}

	Looking back at the introduction, it is possible to say that most of our objectives had been reached and that the principles defined at the beginning had been respected.
	
	The application is active only during the event and can be placed on the Play Store, considering that users cannot login anyway if they do not own the credentials, which are given only to teams' members who subscribe on-line.
	
	Instead of relying on micro transactions from the users, avoiding the pay-per-win typical behaviour, it is offered to organizations (associations, companies, etc.) willingly to prepare and publicize an event in which the game is the main attraction, but not the only one; the most common use case is to prepare a bundle with dinner, game and possibly overnight stay, offered to organized groups of participants up to 30 people for a single event.
	This kind of events is particularly suitable to bring the attention of the people on the restaurant/pub where the dinner takes place or where the players will spend their night, considering that the game is spatially located in the urban area among common citizens.
	
	Just by reading the rules, it is obvious that the game heavy relies on team play, which was one of the aims: pursuing alone the team's objective will likely lead to a loss. Considering that teams are pre-formed, they are likely to play cooperatively to get higher chances of winning. A possible prize shall convince those players which only put efforts in a game when they gain something from it to play seriously.
	
	The relatively short duration of the game increases the perceived value of the time spent on it; the area limitation, instead, assures a continuous interaction between the players which is one of the basic aspect keeping the match exciting and alive. Moreover, the fact that the board can be generated dynamically given the centre and radius, allows to replicate the experience on different scales and cities: the spatial limitation is just related to the board, but the game can virtually take place everywhere, making it different every time the location changes. \\
	
	Given the absence of testing data, it is hard to identify which mechanics should be changed, but even without feedbacks some things can be recognized as sources of probable problems:
	\begin{itemize}
		\item The same principle which excludes the game from the pay-per-win group reduces the possible gain over the application, given that it is a support to the event instead of an event itself and the target niche is becoming more and more populated (even if not many competitors can claim a AR experience into their events);
		\item Some parts of the application work well, but are not modular as they should be; this results in slightly messy code, especially \lstinline|GeoFragment| class in which visibility rules and other features seem to be inherently bound the one to the other, causing some problems from code cleanness point of view;
		\item The location-based AR is still the best choice for games which take place in a large physical area, but are less reliable when it comes to interact between users; a markerless tracker module which works side-by-side to it can help to increase the precision;
		\item It has to be noted that the game is currently thought to run on Android phones since API 16 (Jelly Bean), if we change the hypothesis to later cellphones (e.g., API 21 onward) the battery drain constraint can be relaxed opening up much more possibilities on game mechanics and length;
		\item Some objectives will likely need to be balanced based on a few games try, especially the one which is fulfilled when the game reaches a certain turn and the one which requires reaching a specific amount of money in the team reserve.
	\end{itemize}
	
	This project can be extended by far in numerous fields, after finishing the currently under-construction components; here some are presented, ordered by difficulty of implementation.
	
	\paragraph{Website, team and players registration - easy}
		
	Currently teams and players can be added only via the Firebase dashboard. A website where players can subscribe to the system and the teams can be registered is obviously needed; the only reason it has not been done yet is that the thesis focus is on the mobile part.
	
	\paragraph{Multi-match support - easy}
	
	The game structure is easily scalable to support multiple game instances at the same time. This will impact on the Firebase Database structure and will need to develop an endpoint service capable of managing more than one instance together. Luckily this problem was kept in mind from the beginning of the development and single game instances are structured as a class on their own, running on a dedicated thread.
	
	\paragraph{Game replay - easy}
	
	Adding the replay feature is simple having an already working logging system, even if it can take some time. It is just a matter of creating a class capable of reading the log file and graphically displaying the data contained in it on a map. This tool can be particularly useful to analyse units behaviour, their interaction and check if someone cheated. A real-time replay of the match (which goes on while the game progress but can also be browsed going back and forth to see again particular moments), like the one available in the majority of current e-sports (League of Legends, DotA2, Starcraft) is vital to the event organizers and, eventually, an audience watching the game.
	
	\paragraph{Service-based location update and alert notifications - medium}
	
	The current players' position is known only when they access the application and this is a big problem, because the players cannot always have it active and the GPS system take some time to reach a useful accuracy. This leads to wrong position information both in the map and in the augmented part, which influences the decision making of players and the interaction with them. If, for example, a user stops using the application (which means he is not connected any more and there are no GPS updates), an assassin in AR mode could find his marker standing still in the middle of nothing, target it (because for the game the unit is still there) and then kill him. The unit, when activating again the application will find himself assassinated without a single warning and for no real reason. An enhancement in this area is critical to make the game actually playable. The best solution is to separate from the application the GPS position retrieve and data updates to the server, putting it in a service always active which can then be accessed by all the activities who need to know the current position of the user.
	This would solve many problems, because instead of initializing one or more new location requests for every activity, we would have a single module doing it for all the application (and thus, it can be bound with the application general context extending the custom \lstinline|Application| class).
	The downside is the higher battery consumption, but it can be addressed decreasing the GPS requests rate when the user is not on-line.
		
	This service (or another similar one) would manage push notifications, reporting the events happening in game since last access and showing the timer of the current phase in the status bar.
		
	Obviously the service would die when the game status is changed to \emph{INACTIVE}, avoiding memory leaks.
	
	\paragraph{iOs porting - hard}
	
	Even if Android market share is huge and keep increasing, if the application works only on Android devices it  cuts off a lot of users using iOs operative system: a porting is required. Windows phone market share is actually so small that it makes no sense to port the game on it.
	
	\paragraph{Full game implementation - hard}
		
	The current game implementation is the simplest of all the described possibilities; developing the full version with leaders, a web application to follow units actions and other dropped features is a difficult and long task. Moreover, an evaluation must be done to see if the full version of the game is actually appealing or if the simpler one is already too complex for people to use it.
	
	\paragraph{Automatic and random board creation - hard}
	
	As already discussed in \autoref{focus:board}, the initial aim was to provide a board which adapts to the location, using main streets to define the perimeter and generating irregular shaped zones (but all with more or less equivalent area) with the power-ups randomly displaced. This kind of enhancement can easily turn into another thesis project and can be released unbound from the application, as a stand-alone library to use with Google Maps SDK.
	
	\paragraph{VoIP communication service, drop of text-based messenger - hard}
	
	Another feature that could be the subject for a thesis by its own is the development of a better communication system capable of superseding the textual messenger. Considering the need of rapid decisions and high coordination required to play the game, a VoIP system with automatic voice recognition or push-to-talk option would be much better and faster than sending text messages, especially if it could be accessible in background also when the application is closed. Currently some options are available to start with, needing to install stand-alone clients, which are the porting of the main systems used on desktop environment (Curse, Teamspeak, Discord); integrating one of them into the game or develop one from scratches will improve greatly the game experience.