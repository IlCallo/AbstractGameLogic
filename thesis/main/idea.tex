\chapter{Idea}

	The initial thesis goal was to produce a new AR framework, but after some searches it became pretty much obvious that the current problem with this kind of frameworks is the inactivity rather than their shortage.
	So I switched to the production of an AR game, choosing an existing framework and updating/fixing it for my needs.
	
	\section{Serious Games}
	
		\begin{quoting}
			Games may be played seriously or casually. We are concerned with serious games in the sense that these games have an explicit and carefully thought-out educational purpose and are not intended to be played primarily for amusement. This does not mean that serious games are not, or should not be, entertaining.~\cite{abt:serious}
		\end{quoting}
	
		The first proposed idea was to build up a so called Serious Game, in particular an AR app that could serve as support to civil protection courses during their on-the-field simulations.
		The app structure closely resembled RTS (Real Time Strategy) games: one person, identified as a strategist, had a centralized control of the environment via a web app and could form groups of persons, identified as units with a specified role, and assign those groups to a mission.
		Each role granted to the unit whom had it to perform one or more particular actions, while missions were no more than a sequence of predefined POIs to reach with associated actions to execute nearby.
		
		The idea was interesting, but after defining the basic game rules the project shifted to something more related to the entertainment area due to personal interest, given also the lack of a direct contact with someone who could actually use and test the project after it's completion.
	
	\section{Discworld - Ankh-Morpork}
	
		Searching for ideas in the field of entertainment, I was suggested to chose an already existing board game and to transpose it's rules in an AR mobile game.
		Excluding classic games (Monopoly, Risk), the focus fell on territorial control board games following the trend set up by Ingress and Pokémon GO.
		\textbf{Discworld - Ankh-Morpork} was chosen, a game set in Terry Pratchett's Discworld setting, published by TreeFog Games, and with some particular attributes.
		
		The game setting is about a city in which the patrician just disappeared, leaving a power vacuum that different lords and prominent figures are willingly to fill.
		Two to four players represent the these personalities, everyone with a different victory goal, whom want to control the city.
		
		The game is turn based and controlled by cards which grant particular actions.
		Every player control his minions moving them around on the board (the city itself divided in neighbourhoods), assassinating enemy minions and building palaces to increase its control over certain zones (gaining power-ups).
		It's possible to construct a palace only if the player previously collected enough gold, with some cards or power-up, to pay the needed taxes.
	
		Zones chaotic status is controlled by trouble markers that enable or disable certain actions: you cannot assassinate minions if the zone is calm and you cannot build a palace if the zone is chaotic. Trouble markers are added or removed based on minions movements: every time a minion enters a zone with other minions in it, the trouble marker is added in that zone, every time it exits (or is removed from the board), the trouble marker is removed.
		To bring in more suspense, a random event is fired every once in a while, creating havoc in everyone plans.
		
		Game mechanics are simple and understandable: reading the manual is enough to understand them without the need of a trial match, this does not happen often in this kind of board games.
	
	\section{Game type}
	
		As we have seen in \autoref{soa:games}, nearly all of the currently widespread AR games have been thought to be MMOG (Massive Multiplayer Online Games), based on the assumption that the user must be able to play whenever he wants and wherever he wants, in order to maximize the usage of the app and, indirectly, the revenue.
		The only analysed app that tries to offer a slightly different gaming paradigm is Father.IO which offer the feature to organize a laser tag, instead of just entering it on-the-fly when the user wants.
		
		While putting the user desires first is a legitimate choice, this project focus on another type of games.
		In fact, an aspect which has been tried to keep while transitioning from a board game to an AR game was the idea that lays at its base.
		With Pokémon GO and similar games, the basic idea is that when you does not have anything else to do, you play to escape boredom.
		With many boardgames (we are not talking about casual boardgames here), the idea is different: you decide to invest your time and spend efforts to organize a meet-up with people with, or against which, you want to compete.
		This kind of games stimulate your mind to resolve complex problems and to be adaptive, instead of becoming dull while waiting for the next commitment of the day. \emph{That} is the idea on which this project bases itself.
		
		Some efforts in this direction already exist: the second edition of MoM (Mansion of Madness) has been released in October with an associated mobile application which manages board and story, a duty previously assigned to one of the players, while other boardgames publishers are preparing to follow the same path.
		Walking the same trail, the application developed for this thesis has been thought as an assistant to a board game, except that the board in this case is placed in the real world.
		
		A concept must be clear: this game is half-way between a Zombie Run and an organized evening event.
		Participation won't be for free and will be an all-included package: dinner, game, possibly a prize for the winner.
		The registration will be done several weeks before the event takes place and the participants will be strongly motivated to play and win (they spent for doing so), in this way the risk of people not really motivated to play, that could ruin the game experience to others, is reduced.
	
	\section{Building up the team}
		
		As a programmer, there are some fields in a project like this that requires competences I do not own.
		Also, for personal aim, I never realize projects that do not produce a result and which cannot be used in some way by others: in my mind that's an highly inefficient way of spending time on something.
		
		Following this principle, the obvious way to obtain an high level product from this thesis was to compose a team that could overcome my deficiencies.
		The first thing I searched was someone, or something, to which I could hand the product once completed: it's licit to say that I searched for a problem to which my thesis was the solution.
		I found a good one into PassaPorta\footnote{More informations at their web-site: \url{ https://associazionepassaporta.wordpress.com/}}, an association active in Reggio Emilia which do entertainment both for kids and adults.
		In the latter case, it's mostly about thematic evenings, and this was exactly what I was looking for.
		From PassaPorta I contacted Daniele Barozzi, who helped me to understand how to manage an event with 20-25 participants and which kind of limitations would be needed in terms of game setting and mechanics.
		
		The second thing I needed was someone with a good knowledge in game design, in order to be sure that the game would have been balanced, enjoyable and with rules strong enough to hold possible exploit attempts.
		For this, more or less following the same principle that leads to hackers being employed in anti-virus companies, I contacted Valerio Catellani, known for being the one who study and exploit the hidden mechanics in every game in order to win in an unexpected, and most of the time hateful, way. His job had been to help me transpose the board game mechanics to the AR application and stress test them in limit scenarios to be sure no one could cheat.
		
		Third, a graphic was required, to help me define the game UX, UI and create mock-ups. Half my family helped me out for this: Francesca for general design definition and paper-based icons, Caterina for mock-up preparation and Luca for vectorial images.
		
		Last group of competences I lacked was the storytelling: a background story was needed, established that we could not use Discworld setting, because covered by copyright, and we could not even ask permission to the game publisher because his license to use that setting expired. Both Daniele Barozzi and my brother, Luca, helped me out in this, writing a background story that could fit the rules we previously defined and at the same time be appealing and coherent.