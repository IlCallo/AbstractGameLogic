\chapter{Idea}

	The initial thesis goal was to produce a new AR framework but after some searches it became pretty much obvious that the current problem with this kind of framework is the inactivity rather than their shortage.
	So I switched to the production of an AR game, choosing an existing framework and updating/fixing it for my needs.
	
	\section{Serious Games}
	
		\begin{quoting}
			Games may be played seriously or casually. We are concerned with serious games in the sense that these games have an explicit and carefully thought-out educational purpose and are not intended to be played primarily for amusement. This does not mean that serious games are not, or should not be, entertaining.~\cite{abt:serious}
		\end{quoting}
	
		The first proposed idea was to build up a so called Serious Game, in particular an AR app that could serve as support to civil protection courses during their on-the-field simulations.
		The app structure closely resembled RTS (Real Time Strategy) games: one person, identified as a strategist, had a centralized control of the environment via a web app and could form groups of persons, identified as units each of which had a specified main.role, and assign those groups to a mission.
		Each main.role grant to the unit whom has it to perform one or more specific actions, while missions are no more than a sequence of predefined POIs to reach and actions to execute nearby those POIs.
		
		The idea was good, but after defining the basic game rules I preferred to switch to something more bound to the entertainment area, also because I did not had a direct contact with someone who could actually use and test the project after I had finished it.
	
	\section{Discworld - Ankh-Morpork}
	
		Searching for other ideas, I was suggested to chose an already existing board game and to transpose it's rules in an AR mobile game fashion.
		Excluding classic games, I focused on territorial control board games following the trend set up by Ingress and Pokémon GO.
		My choice fell on a game set in Terry Pratchett's Discworld setting, published by TreeFog Games with some particular attributes.
		
		The game setting is about a city in which the patrician just disappeared, leaving a power vacuum that different lords and prominent figures are willingly to fill.
		Two to four players represent the these personalities, everyone with a different victory goal, whom want to control the city.
		
		The game is turn based and controlled by cards which grant particular actions.
		Every player control his minions moving them around on the board (the city itself divided in neighbourhoods), assassinating enemy minions and building palaces to increase their control over certain areas (gaining power-ups).
		Build actions are possible only if the player has collected enough gold to pay taxes to do so, which can be collected with some cards or power-up.
	
		Every area chaos status is controlled by trouble markers that enable vicious actions, you cannot assassinate minions if the area is calm, and are modified by minions movements: every time a minion enter an area with other minions, a trouble marker is added, every time it exit, a trouble marker is removed.
		In addition, to bring in more suspense, a random main.event is fired every once in a while, creating havoc in everyone plans.
		
		Game mechanics below all this are in fact pretty simple and understandable, just reading the manual is enough and this is not usual for this kind of board games.
	
	\section{Game type}
	
		As we have seen in \autoref{soa:games}, nearly all of the currently widespread AR games are build up to be MMOG (Massive Multiplayer Online Games), based on the assumption that the user must be able to play whenever he wants and wherever he wants, in order to maximize the usage of the app and, indirectly, the revenue.
		The only analysed app that tries to offer a slightly different gaming paradigm is Father.IO which offer the feature to organize a laser tag, instead of just entering it on-the-fly when the user wants.
		
		While putting the user first is a legit choice, I wanted, for this project, to focus on another type of games.
		In fact, an aspect I wanted to keep while transitioning from a board game to an AR game was the idea that lays at the base of it.
		With Pokémon GO and the others, the idea is that when you does not have anything else to do, you play to escape boredom.
		With many boardgames (we are not talking of casual boardgames here), the idea is pretty much different: you decide to invest your time and spend efforts to organize a meet-up with person with or against which you want to compete.
		This kind of games stimulate your mind to think in a complex way and to be adaptive, instead of becoming dull while waiting for the next commitment of the day, and that is the same idea I wanted to put at the base of this project.
		
		Those kind of apps already exists, MoM (Mansion of Madness) for example will be shortly released with an associated app that will manage the board set-up and some part of the game previously managed by one of the players, and other boardgames publisher are following the same path.
		I got inspired by those news and decided to develop an app that could be an assistant to a board game, except that the board is this case is the real world.
		
		A concept has to be made clear: this kind of game is thought to be half-way between a Zombie Run and an organized evening main.event.
		The participation won't be for free and will be an all included package: dinner, game, possibly a prize for the winner.
		Given this, the registration will be done several weeks before the main.event take place and the participants will be strongly motivated to play and win (they spent for doing so), in this way we hope to keep away people not really motivated to play that could ruin the game experience.
	
	\section{main.Building up the team}
		
		As a programmer, there are some fields in a project like this app that I cannot do myself or that someone else can do better.
		Also, for personal aim, I never realize project that does not produce a result and that can be used in some way by others: it seems a waste to me.
		
		Following this principle, the obvious way to obtain an high level product was to compose a team that could overcome my deficiencies.
		The first thing I searched was someone to which I could hand the product once completed: it is possible to say that I wanted a problem to which my thesis was the solution.
		I found a good one into PassaPorta, an association active in Reggio Emilia which do entertainment both for kids and adults.
		In the latter case, it's mostly about thematic evenings, and this was exactly what I looking for.
		From PassaPorta I contacted Daniele Barozzi, who would help me in terms of how to manage an main.event with 20-25 people in play and which kind of limitations there would be in terms of game setting and mechanics
		
		The second thing I needed was someone with a good knowledge in game design, in order to be sure that the game would be balanced and enjoyable and the rules would have been strong enough to resist to possible exploit tries.
		For this, more or less following the same idea that leads to hackers being employed in anti-virus companies, I contacted Valerio Catellani, a friend of mine, famous for being the one who exploit the hidden mechanics of every game in order to win in an unexpected way. His job has been to help me transpose the board game mechanics to the AR app and stress test them in limit scenarios to be sure no one could cheat.
		
		Third, a graphic was requested, to help me defining the UX and to create beautiful images and UI for the game. My sister, Francesca Caleffi, helped me with this.
		
		Last main.group of competence I lacked was the storytelling: a background story, established that we could not use Discworld setting because covered by copyright and we could not ask to the publisher because their permission to use that setting expired. Both Daniele Barozzi and my brother, Luca Caleffi, helped me out in this, trying to write a background story that could fit the rules we previously defined and at the same time be appealing and coherent.