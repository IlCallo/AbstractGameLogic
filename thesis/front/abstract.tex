\sommario
	
	This document describes the ideation, design and implementation of a game heavily relying on the AR (Augmented Reality) concept; its proposed use case is to support structured recreational events composed by dinner, a match and other boundary activities; the game code had been thought to be setting-independent and with minor alteration (mainly graphic ones) can be adapted to different scenarios.
	
	The game infrastructure is composed by an Android application and a server endpoint, written in Java, which manages the game work flow and phase changes.
	Authentication and data management rely on Firebase service.
	
	Firebase data model is explained in detail; the Java one is covered only to highlight the most interesting pieces of code and differences between server and mobile application, considering that the Android data model is a heavy simplification of the Java one.
	
	A brief explanation of the existent AR technologies is provided, as well as a description of the features of the main frameworks which had been browsed and why the BeyondAr one had been chosen.
	
	Some of the most famous AR mobile applications and structured recreational events are mentioned highlighting their connection with AR basic ideas, to show where this project locates itself in the current panorama.
	
	A selection of the main problems met during the development is provided, explaining how they've been solved and, where needed, through which path the solution had been reached.
	
	Lastly, the produced project is reviewed and checked against the initial objectives, which have been mostly accomplished even if some modules are currently incomplete, and possible upgrades are proposed, some of which can be the starting point for new thesis.